%Optics Homework_5
\documentclass[10pt,a4paper]{article}
\usepackage[UTF8]{ctex}
\usepackage{bm}
\usepackage{amsmath}
\usepackage{amssymb}
\usepackage{graphicx}
\title{光学作业\#5}
\author{陈稼霖 \and 45875852}
\date{2018.11.22}
\begin{document}
\maketitle
\section*{3-23}
\subsection*{(1)}解:
$M_1$移动的距离为
\[
\Delta h = N\frac{\lambda}{2} = 2.9465\mu m
\]
\subsection*{(2)}解:
设开始时中心亮斑的干涉级为$k$,视场角范围为$\theta$。移动平面镜前,视场中心亮环对应的光程差为
\[
\Delta L_0 = 2h = k\lambda
\]
视场边缘亮环对应的级数为$(k-12)$,光程差为
\[
\Delta L_1 = 2h\cos\theta = (k-12)\lambda
\]
移动平面镜后,视场边缘亮环对应的级数为$(k-10-15)$,光程差为
\[
\Delta L_2 = 2(h-\Delta h)\cos\theta = (k-10-5)\lambda
\]
以上三式联立解得
\[
k = \frac{24\Delta h}{2\Delta h-3\lambda} \approx 17
\]
故开始时中心亮斑的干涉级为$17$。
\subsection*{(3)}解:$M_1$移动后,从中心向外数第$5$个亮环的干涉级为$k-10-5 \approx 2$。
\section*{3-25}解:
干涉条纹由最清晰到最模糊的过程中光程差的变化值为
\[
\Delta L= \frac{\lambda^2}{2\Delta\lambda} = N\lambda
\]
解得钠双线的两个波长之差为
\[
\Delta\lambda = \frac{\lambda}{2N} = 0.6nm
\]
故钠双线的两个波长分别为
\begin{align*}
&\lambda_1 = \lambda+\frac{\Delta \lambda}{2} = 589.6nm\\
&\lambda_2 = \lambda-\frac{\Delta \lambda}{2} = 589.0nm
\end{align*}
\section*{3-27}解:
这台仪器测长精度为
\[
\delta l = \delta N\frac{\lambda}{2} = 32nm
\]
一次测长量程即空间相干长度的二分之一,为
\[
L = \frac{\lambda^2}{2\Delta\lambda} = 2.00m
\]
\section*{3-28}
\subsection*{(1)}解:
迈克尔逊干涉仪的吞(吐)一圈条纹镜面所要移动的距离为
\[
l = \frac{\lambda}{2}
\]
电信号的周期即为镜面移动以上距离所需的时间
\[
T = \frac{1}{\nu} = \frac{l}{v} = \frac{\lambda}{2v}
\]
解得入射光的波长为
\[
\lambda = \frac{2v}{\nu}
\]
\subsection*{(2)}解:
反射镜平移的速度为
\[
v = \frac{\lambda\nu}{2} = 15\mu m/s
\]
\subsection*{(3)}解:
钠黄光双线的电信号频率分别为
\begin{align*}
&\nu_1 = \frac{2v}{\lambda_1}\\
&\nu_2 = \frac{2v}{\lambda_2}
\end{align*}
产生的电信号的拍频为
\[
\Delta\nu = \nu_1 - \nu_2 = 2v(\frac{1}{\lambda_2} - \frac{1}{\lambda_2}) \approx \frac{2v\Delta\lambda}{\lambda^2}
\]
代入$\Delta\lambda = \lambda_1-\lambda_2 = 0.6nm, \lambda = \frac{\lambda_1+\lambda_2}{2} = 589.3nm$,解得电信号拍频为
\[
\Delta\nu = 5.2\times10^{-2}Hz
\]
\section*{3-29}解:
法布里-珀罗干涉仪的色分辨本领为
\[
\frac{\lambda}{\delta\lambda} = \pi k\frac{\sqrt{R}}{1-R}
\]
干涉级数和间距之间的关系为
\[
2nh = k\lambda
\]
以上两式联立解得所需间隔$h$为
\[
h = \frac{1-R}{2n\pi\sqrt{R}}\frac{\lambda^2}{\delta\lambda} = 2.94cm
\]
\section*{3-31}
\subsection*{(1)}解:
干涉级数和间距之间的关系为
\[
2nh\cos\theta = k\lambda
\]
中心处$\cos\theta = 1$,解得中心干涉级数为
\[
k = \frac{2nh}{\lambda} \approx 1.7\times10^5
\]
\subsection*{(2)}解:
在倾角为$1^\circ$附近干涉环的半角宽度为
\[
|\Delta i| = \frac{\lambda}{2\pi nh\sin i}\frac{1-R}{\sqrt{R}} = 2.2\times10^{-6}rad = 0.45''
\]
\subsection*{(3)}解:
该法-珀腔分辨谱线可分辨的最小波长间隔为
\[
\delta\lambda = \frac{\lambda}{\pi k}\frac{1-R}{\sqrt{R}} = \frac{\lambda^2}{2n\pi h}\frac{1-R}{\sqrt{R}} = 2.3\times10^{-14}m = 2.3\times10^{-5}nm
\]
色分辨本领为
\[
\frac{\lambda}{\delta\lambda} = 2.6\times10^7
\]
\subsection*{(4)}解:
法-珀腔透射率公式为
\[
\eta_T = \frac{1}{1+\frac{4R\sin^2(\delta/2)}{(1-R)^2}}
\]
其中相邻光线之间的相位差为
\[
\delta = \frac{4\pi nh}{\lambda}
\]
透射率达到峰值的条件为
\[
\frac{2\pi nh}{\lambda} = k\pi, ~~~~k = \pm1,\pm2,...
\]
解得透射率达到峰值的光线波长为
\[
\lambda = \frac{2nh}{k}
\]
可见光波长范围为$[400nm, 760nm]$,其对应的干涉级数满足
\[
400nm < \frac{2nh}{k} < 760nm
\]
解得干涉级数范围
\[
1.3\times10^5 < k < 2.5\times10^5
\]
故透射最强的谱线有$2.5\times10^5 - 1.3\times10^5 = 1.2\times10^5$条。每条谱线宽度为
\[
\Delta\lambda_k = \frac{\lambda}{\pi k}\frac{1-R}{\sqrt{R}}
\]
代入中间波长$\lambda = \frac{400 + 760}{2}nm = 580nm$,中间谱线级数$k = 1.9\times10^5$,解得每条谱线宽度为
\[
\Delta\lambda = 1.9\times10^{-14}m = 1.9\times10^{-5}nm
\]
\subsection*{(5)}解:
对谱线波长公式
\[
\lambda = \frac{2nh}{k}
\]
两边取微分再除以原式得到得
\[
\frac{\delta\lambda}{\lambda} = \frac{\delta h}{h} = 10^{-5}
\]
故谱线漂移量(相对值)与腔长改变量(相对值)相等,为$10^{-5}$。
\section*{3-32}
\subsection*{(1)}
透射谱线波长公式为
\[
\lambda = \frac{2nh}{k}
\]
可见光波长范围为$[400nm, 760nm]$,其对应的干涉级数满足
\[
400nm < \frac{2nh}{k} < 760nm
\]
解得干涉级数范围
\[
1.6 < k < 3.1
\]
故在可见光范围内,透射最强的谱线有2条。
\subsection*{(2)}解:
对于第$k=2$级谱线,其波长为
\[
\lambda_2 = \frac{2nh}{2} = 620nm
\]
其谱线宽度为
\[
\Delta\lambda_2 = \frac{\lambda_2}{2\pi}\frac{1-R}{\sqrt{R}} = 4.03nm
\]
对于第$k=3$级谱线,其波长为
\[
\lambda_3 = \frac{2nh}{3} = 413nm
\]
其谱线宽度为
\[
\Delta\lambda_3 = \frac{\lambda_3}{3\pi}\frac{1-R}{\sqrt{R}} = 1.79nm
\]
\end{document}
